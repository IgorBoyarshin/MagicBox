\documentclass[12pt]{article}
%\usepackage{amsmath}
\usepackage{mathtools}
\usepackage[a4paper, margin=1in]{geometry}
\usepackage{algpseudocode}
\usepackage{graphicx}
\graphicspath{ {./} }
%\usepackage{hyperref}
\usepackage[utf8]{inputenc}
\usepackage[english, russian]{babel}
\usepackage[T1, T2A]{fontenc}
\usepackage{indentfirst} % do indent the first line of a section (including sets to yes)
\usepackage{array}

\newcolumntype{x}[1]{%
>{\centering\hspace{0pt}}p{#1}}%

\begin{document}

\section{Вступ}

З розвитком інформаційних технологій зростає використання віддаленого надання різноманітних обчислювальних послуг.
Значною мірою це стало можливим завдяки швидкому розповсюдженню хмарних обчислень, які стають все більш доступними.

Ефективність віддаленого надання обчислень напряму залежить від ефективності ідентифікації абонента в системі.
Питання ефективності ідентифікації абонента, в свою чергу, зводиться до знаходження компромісу між швидкістю і зручністю цієї ідентифікації для користувача та надійністю такого каналу зв'язку.

Через різке зростання кількості користувачів важливим стає питання швидкості роботи засобів ідентифікації абонентів, адже швидкість опрацювання одного абонента має прямий вплив на швидкість роботи системи загалом.
Оскільки значна частина сервісів надання віддалених обчислення є комерційною, то одним з передових критеріїв при створенні системи ідентифікації стає зручність та легкість роботи з нею для користувача.

У сучасних умовах до системи ідентифікації абонентів значно зростають вимоги надійності, зокрема на передній план виходить криптографічно-строга ідентифікація.
Аналіз відомих атак на системи ідентифікації показав, що найбільш ефективним засобом протидії їм є повторна ідентифікація  абонента в системі з певним періодом.

Таким чином, створення криптографічно-строгих та швидких методів ідентифікації абонента в системі є актуальною задачею на сьогоднішньому етапі розвитку комп'ютерних технологій.

\section{Опис структури}

Одним із варіантів досягнення поставленої мети --- кардинальне прискорення строгої ідентифікацій --- є перехід до альтернативного алгебраїчного базису, зокрема булевих функціональних перетворювачів.

Нелінійні булеві функціональні перетворювачі мають важливу для криптографічних застосувань властивість незворотності. Тобто вони практично унеможливлюють віднаходження вхідного коду по вихідному при аналітично- чи процедурно-заданому перетворенні.

Разом з тим обчислення булевих функціональних перетворювачів вимагає значно менше обчислювальних ресурсів в порівнянні з використанням інших алгебр, особливо при апаратній реалізації.
Загальновизнаним є той факт, що при однаковому рівні захисту використання булевих функцій дозволяє прискорити обчислення в 3-4 рази.

Завдяки цьому булеві функціональні перетворювачі знайшли широке застосування в сучасних засобах криптографічного захисту інформації: алгоритмах симетричного шифрування, хеш-перетворюваннях та потокових шифрах.

Аналіз показав, що для вирішення задачі ефективної ідентифікації може бути використано хеш-перетворення з програмованими колізіями. Такий підхід передбачає визначення колізій для заданого вихідного коду безпосередньо в процесі проектування.

Таким чином, постає задача синтезу баготозначної зворотньої функції.

При розробці такого функціонального перетворювача висуваються наступні вимоги:
\begin{enumerate}
\itemsep=0em
\item довжина вхідного вектору має бути більшою або рівною довжині вихідного
\item структура функціонального перетворювача має забезпечувати залежність кожного розряду вихідного коду від всіх розрядів вхідного
\item для захисту від лінійного криптоаналізу перетворення мають бути нелінійні
\item структура перетворювача має бути простою та допускати ефективну програмну та апаратну реалізацію
\item вважаючи на реальні розрядності вхідних та вихідних кодів, функціональний перетворювач має представлятися в процедурній формі
\item можливість формування в рамках однієї структури великої кількості різних функціональних перетворювачів
\end{enumerate}

Поставленим вимогам задовольняє структура, показана на рис 1.

\begin{center}
\textbf{рис 1}
\end{center}

Для цієї структури прийняті такі умовні позначення:
\begin{enumerate}
\itemsep=0em
\item $k$ -- розрядність фрагмента;
\item $n$ -- довжина вихідного вектора;
\item $m = \frac{n}{k}$ -- кількість фрагментів вихідного вектора;
\item $p = 2 \cdot m - 1$ -- кількість фрагментів вхідного вектора;
\item $s = k \cdot p$ -- довжина вхідного вектора;
\item $X = \{x_1, x_2 \ldots x_p\}$ -- вхідний вектор, $\forall i=1,2 \ldots p : x_i \in \{0,1, \ldots ,2^k-1\}$;
\item $Y = \{y_1, y_2 \ldots y_m\}$ -- вихідний вектор, $\forall i=1,2 \ldots m : y_i \in \{0,1, \ldots ,2^k-1\}$;
\end{enumerate}

Будь-який функціональний перетворювач (ФП) можна представити у вигляді таблиці з $2^k$ рядків. Тоді значенням на вході до ФП буде називатися адреса комірки з цієї таблиці, а значенням на виході з ФП - числове значення цієї комірки.

В подальшому описі через $v_{i,j}$ позначено значення на вході j-го ФП на i-му шарі, а через $w_{i,j}$ - значення на виході: $f_j(v_{i,j}) = w_{i,j}$.

Функціональний перетворювач формує n-розрядний вихідний ідентифікувальний код користувача з вхідного паролю Х.

\quad \\

Запропонований метод строгої ідентифікації включає в себе дві процедури:
\begin{enumerate}
\itemsep=0em
\item реєстрація користувача в системі
\item сеанс ідентифікації користувача в системі
\end{enumerate}

Процедура реєстрації користувача складається з послідовного виконання наступних дій:
\begin{enumerate}
\itemsep=0em
\item користувач надсилає запит на реєстрацію в системі і передає їй свій відкритий зачиняючий ключ
\item система формує ідентифікувальний код користувача $Y$, шифрує його відкритим зачиняючим ключем користувача та відправляє користувачу зашифрований код $Y$
\item користувач отримує зашифрований $Y$ та відчиняє його своїм закритим відчиняючим ключем
\item користувач генерує перетворення по описаному нижче методу і отримує набір векторів $X$
\item отримані вектори $X$ користувач зберігає у себе в захищеній пам'яті\\
\item перетворення шифрується відкритим закриваючим ключем системи та відправляється до системи
\item  система відновлює перетворення з використанням свого секретного відчиняючого ключа та зберігає перетворення в пам'яті
\item система створює та очищую список використаних паролів користувача
\end{enumerate}

Процедура ідентифікації користувача для $q$-го сеансу зводиться до виконання наступних дій:
\begin{enumerate}
\itemsep=0em
\item користувач вибирає черговий пароль $X_q$ зі свого списку паролів
\item користувач передає сеансовий пароль $X_q$ до системи
\item система приймає сеансовий пароль $X_q$ та виконує перетворення з отриманням вихідного $Y'$
\item система виконує порівняння ідентифікатора користувача  $Y$ з отриманим $Y'$. Якщо вони однакові і сеансовий пароль $X_q$ відсутній у списку використаних паролів користувача, то користувачу надається право доступу до системи, а сеансовий пароль $X_q$ додається до списку використаних паролів користувача. Інакше користувачу доступ до системи не надається
\end{enumerate}

\section{Алгоритм}

Вихідний вектор Y вважається заданим.

Пропонований метод передбачає наступну послідовність дій для формування векторів Х:

\begin{enumerate}
\itemsep=0em
\item Всі комірки всіх ФП заповнюються значенням $-1$, що відповідає невизначеному стану: $\forall j=1,2 \ldots 2 \cdot m-1, l=0,1 \ldots 2^k-1 : f_j(l) = -1$.
\item В якості поточного шару вибрати останній: $i=m-1$.
\item Для кожного ФП$_j$ в останньому шарі $(j=1,2 \ldots m+1)$ знаходиться адреса комірки $v_{m-1,j}$, для якої стан ФП$_j$ невизначений.
\item Перейти до попереднього шару: $i=i-1$.
\item В першому ФП$_1$ шукається адреса $v_{i,1}$ комірки, для якої стан ФП$_1$ визначений: $f_1(v_{i,1}) \neq -1$;
якщо такої комірки немає, то адреса $v_{i,1}$ і відповідне значення $w_{i,1}$ комірки вибираються з використанням генератора випадкових чисел (але при цьому $v_{i,1}$ має відрізнятися від вибраного в пункті 3 $v_{m-1,1}$: $v_{i,1} \neq v_{m-1,1}$), а до ФП$_1$ записується $f_1(v_{i,1})=w_{i,1}$.
\item Для решти ФП$_j$ в поточному шарі $(j=2,3 \ldots 2 \cdot m-i)$ послідовно визначаються значення на виході $w_{i,j}$ за такою формулою: $w_{i,j} = w_{i,j-1} \oplus v_{i+1,j-1}$;
при цьому адреса $v_{i,j}$ комірки ФП$_j$ вибирається з використанням генератора випадкових чисел серед тих, для яких значення комірки визначене і дорівнює знайденому $w_{i,j}$: $f_j(v_{i,j})=w_{i,j}$.
якщо комірок зі значенням $w_{i,j}$ у ФП$_j$ немає, то береться будь-яка комірка $v_{i,j}$ з невизначеним станом (але при цьому $v_{i,j}$ має відрізнятися від вибраного в пункті 3 $v_{m-1,j}$: $v_{i,j} \neq v_{m-1,j}$) і її значення визначається знайденим $w_{i,j}$: $f_j(v_{i,j})=w_{i,j}$.
\item Якщо номер поточного шару $i \geq 2$, тобто поточний шар не є першим, то здійснюється перехід на повторне виконання пункту 4.
\item Значення компонентів $q$-го вхідного вектора $X_q$ визначаються як значення на вході відповідних ФП в першому шарі: $x_{q,j}=v_{1,j}$ $(j=1,2 \ldots 2 \cdot m-1)$.
\item Для першого ФП$_1$ в останньому шарі за допомогою генератора випадкових чисел вибирається значення $w_{m-1,1}$ комірки, адреса $v_{m-1,1}$ якої була визначена в пункті 3, і вибране значення записується в ФП$_1$ за відповідною адресою: $f_1(v_{m-1,1})=w_{m-1,1}$.
\item Для решти ФП$_j$ в останньому шарі $(j=2,3 \ldots m+1)$ послідовно знаходяться значення $w_{m-1,j}$ комірок за відповідними адресами $v_{m-1,j}$, що були знайдені в пункті 3: $w_{m-1,j}=(x_{q,j-1} \oplus y_{j-1}) \oplus w_{m-1,j-1}$. Значення ФП$_j$ визначається наступним чином: $f_j(v_{m-1,j}) = w_{m-1,j}$.
\item Для всіх ФП$_j$ $(j=1,2 \ldots 2 \cdot m-1)$ знаходиться кількість невизначених в них комірок $u_j$: $u_j= \ldots$.
\item Якщо кількість невизначених комірок $u_j$ для всіх ФП$_j$ $(j=1,2 \ldots 2 \cdot m-1)$ не менше $m-1$: $\forall j \in \{1,2 \ldots 2 \cdot m-1\}: u_j \geq m-1$, то перейти до формування наступного вхідного вектора $X_q$: $q=q+1$, перехід на пункт 2.
\item Всі невизначені значення всіх ФП заповнити за допомогою генератора випадкових чисел числами в діапазоні $0 \ldots 2^k-1$.
\end{enumerate}

\section{Приклад}

Роботу запропонованого методу ілюструє приклад для структури з параметрами $k=3$ та $m=3$ (рис. \ref{fig:ex1}). %TODO ref
Комірки з невизначеним станом $(-1)$ зображені як пусті комірки.

\begin{figure}[h]
\centering
\includegraphics[width=0.6\textwidth]{ex2}
\caption{Структура з параметрами $k=3$ та $m=3$}
\label{fig:ex1}
\end{figure}

Нехай вектор $Y$ заданий і має значення $Y = \{ 2, 3, 5 \}$.

%+++++++++++++++++++++++++++++++++++++++++++++++++++++++++++++++++
\textbf{Отримання першого вектору $X_1$} виконується наступним чином.
%+++++++++++++++++++++++++++++++++++++++++++++++++++++++++++++++++

Перед початком роботи всі комірки мають невизначений стан (пусті).

Згідно з пунктами 2 та 3 запропонованого методу, в якості поточного шару вибирається останній ($i = m - 1 = 2$), після чого для кожного ФП$_j$ в ньому знаходиться адреса $v_{2, j}$ будь-якої пустої комірки. Нехай були вибрані комірки $v_{2,1}=2, v_{2,2}=1, v_{2,3}=2, v_{2,4}=3$.

Здійснюється перехід до попереднього шару: $i = i - 1 = 1$ (пункт 4).

Відповідно до пункту 5 вибирається будь-яка з комірок, так як у ФП$_1$ є тільки пусті комірки.
Нехай вона має адресу $v_{1,1}=1$, і нехай її значення $w_{1,1}=7$.
До ФП$_1$ записується $f_1(1)=7$.

Згідно пункту 6, за формулою $w_{i,j} = w_{i,j-1} \oplus v_{i+1,j-1}$ послідовно визначається решта значень $w_{1,j}$ в поточному (першому) шарі зліва направо, а саме: $w_{1,2} = 7 \oplus 2 = 5, w_{1,3} = 5 \oplus 1 = 4, w_{1,4} = 4 \oplus 2 = 6, w_{1,5} = 6 \oplus 3 = 5$.
При цьому, так як відповідні ФП$_j$ мають тільки пусті комірки, то адреси $v_{1,j}$ для них вибираються з використанням генератора випадкових чисел, наприклад: $v_{1,2} = 2, v_{1,3} = 3, v_{1,4} = 1, v_{1,5} = 2$, а значення цих комірок визначаються відповідно знайденим $w_{1,j}$: $f_2(2)=5, f_3(3)=4, f_4(1)=6, f_5(2)=5$.

Так як поточний шар є першим ($i=1$), то прохід нагору завершується (згідно пункту 7).

Відповідно до 8-го пункту, компоненти вхідного вектора $X_1$ визначаються як значення на вході відповідних ФП в першому шарі: $x_{1,1}=v_{1,1}=1, x_{1,2}=v_{1,2}=2, x_{1,3}=v_{1,3}=3, x_{1,4}=v_{1,4}=1, x_{1,5}=v_{1,5}=2$.

Згідно пункту 9, для ФП$_1$ в останньому шарі випадковим чином вибирається значення комірки, адреса $v_{2,1}=2$ якої була визначена раніше в пункті 3. Нехай це значення $w_{2,1}=4$. До ФП$_1$ записується $f_1(2)=4$.

Відповідно пункту 10, за формулою $w_{m-1,j}=(x_{q,j-1} \oplus y_{j-1}) \oplus w_{m-1,j-1}$ послідовно знаходиться решта значень комірок в останньому шарі, відповідні адреси яких були визначені раніше (в пункті 3): $w_{2,2} = (1 \oplus 2) \oplus 4 = 7, w_{2,3} = (2 \oplus 3) \oplus 4 = 5, w_{2,4} = (3 \oplus 5) \oplus 5 = 3$. Значення відповідних ФП$_j$ визначаються наступним чином: $f_2(1)=7, f_3(2)=5, f_4(3)=3$.

Таким чином, отримано значення вектора $X_1=\{1,2,3,1,2\}$ і завершено один повний прохід методу. Поточний стан ФП показано в таблиці 1 (комірки з невизначеним станом $(-1)$ зображені як пусті комірки). %TODO ref

Так як в кожному ФП залишилося хоча б 2 комірки з невизначеним станом, то можна продовжувати формування наступних векторів $X$.

\renewcommand{\arraystretch}{1.8}
\begin{table}[h]
\begin{flushright}
	\textit{Таблиця 1: стани ФП після 1-го та 2-го проходу}
\end{flushright}
\begin{center}
\begin{tabular}{ | c | c | c | c | c | c | }
	\hline
	\# & \multicolumn{5}{ | c | }{\textbf{Номер ФП}} \\ \cline{2-6}
	& \textbf{1} & \textbf{2} & \textbf{3} & \textbf{4} & \textbf{5} \\ \hline
	\textbf{0} & & & & & \\ \hline
	\textbf{1} & \textbf{7} & \textbf{7} & & \textbf{6} & \\ \hline
	\textbf{2} & \textbf{4} & \textbf{5} & \textbf{5} & & \textbf{5} \\ \hline
	\textbf{3} & & & \textbf{4} & \textbf{3} & \\ \hline
	\textbf{4} & & & & & \\ \hline
	\textbf{5} & & & & & \\ \hline
	\textbf{6} & & & & & \\ \hline
	\textbf{7} & & & & & \\ \hline	
\end{tabular}
\quad
\begin{tabular}{ | c | c | c | c | c | c | }
	\hline
	\# & \multicolumn{5}{ | c | }{\textbf{Номер ФП}} \\ \cline{2-6}
	& \textbf{1} & \textbf{2} & \textbf{3} & \textbf{4} & \textbf{5} \\ \hline
	\textbf{0} & & \textbf{1} & & \textbf{2} & \\ \hline
	\textbf{1} & 7 & 7 & & 6 & \\ \hline
	\textbf{2} & 4 & 5 & 5 & & 5 \\ \hline
	\textbf{3} & & \textbf{2} & 4 & 3 & \\ \hline
	\textbf{4} & & & & & \textbf{7} \\ \hline
	\textbf{5} & \textbf{2} & & \textbf{1} & & \\ \hline
	\textbf{6} & & & \textbf{2} & & \\ \hline
	\textbf{7} & & & & \textbf{7} & \\ \hline	
\end{tabular}
\end{center}
\end{table}
\renewcommand{\arraystretch}{1}

%+++++++++++++++++++++++++++++++++++++++++++++++++++++++++++++++++
\textbf{Отримання другого вектору $X_2$} виконується наступним чином.
%+++++++++++++++++++++++++++++++++++++++++++++++++++++++++++++++++

Згідно з пунктами 2 та 3 запропонованого методу, в якості поточного шару вибирається останній ($i = m - 1 = 2$), після чого для кожного ФП$_j$ в ньому знаходиться адреса $v_{2, j}$ будь-якої пустої комірки. Нехай були вибрані комірки $v_{2,1}=5, v_{2,2}=3, v_{2,3}=5, v_{2,4}=0$.

Здійснюється перехід до попереднього шару: $i = i - 1 = 1$ (пункт 4).

Відповідно до пункту 5 вибирається будь-яка з комірок з визначеним станом у ФП$_1$.
Нехай була вибрана комірка з адресою $v_{1,1}=2$, що має значення $w_{1,1}=4$.

Згідно пункту 6, за формулою $w_{i,j} = w_{i,j-1} \oplus v_{i+1,j-1}$ послідовно визначається решта значень $w_{1,j}$ в поточному (першому) шарі зліва направо, а саме: $w_{1,2} = 4 \oplus 5 = 1, w_{1,3} = 1 \oplus 3 = 2, w_{1,4} = 2 \oplus 5 = 7, w_{1,5} = 7 \oplus 0 = 7$.
При цьому, так як у жодному з відповідних ФП$_j$ немає комірок, що мають такі значення, то адреси $v_{1,j}$ вибираються випадковим чином, наприклад: $v_{1,2} = 0, v_{1,3} = 6, v_{1,4} = 7, v_{1,5} = 4$, а значення цих комірок визначаються відповідно знайденим $w_{1,j}$: $f_2(0)=1, f_3(6)=2, f_4(7)=7, f_5(4)=7$. До ФП записується: $f_2(0)=1$, $f_3(6)=2$, $f_4(7)=7$ та $f_5(4)=7$.

Так як поточний шар є першим ($i=1$), то прохід нагору завершується (згідно пункту 7).

Відповідно до 8-го пункту, компоненти вхідного вектора $X_2$ визначаються як значення на вході відповідних ФП в першому шарі: $x_{2,1}=v_{1,1}=2, x_{2,2}=v_{1,2}=0, x_{2,3}=v_{1,3}=6, x_{2,4}=v_{1,4}=7, x_{2,5}=v_{1,5}=4$.

Згідно пункту 9, для ФП$_1$ в останньому шарі випадковим чином вибирається значення комірки, адреса $v_{2,1}=5$ якої  була визначена раніше в пункті 3. Нехай це значення $w_{2,1}=2$. До ФП$_1$ записується $f_1(5)=2$.

Відповідно пункту 10, за формулою $w_{m-1,j}=(x_{q,j-1} \oplus y_{j-1}) \oplus w_{m-1,j-1}$ послідовно знаходиться решта значень комірок в останньому шарі, відповідні адреси яких були визначені раніше (в пункті 3): $w_{2,2} = (2 \oplus 2) \oplus 2 = 2, w_{2,3} = (0 \oplus 3) \oplus 2 = 1, w_{2,4} = (3 \oplus 5) \oplus 1 = 2$. Значення відповідних ФП$_j$ визначаються наступним чином: $f_2(3)=2, f_3(5)=1, f_4(0)=2$.

Таким чином, отримано значення вектора $X_2=\{2,0,6,7,4\}$ і завершено другий прохід методу. Поточний стан ФП наведено в таблиці 1 (комірки з невизначеним станом $(-1)$ зображені як пусті комірки). %TODO ref

Так як в кожному ФП залишилося хоча б 2 комірки з невизначеним станом, то можна продовжувати формування наступних векторів $X$.

%+++++++++++++++++++++++++++++++++++++++++++++++++++++++++++++++++
\textbf{Отримання третього вектору $X_3$} виконується наступним чином.
%+++++++++++++++++++++++++++++++++++++++++++++++++++++++++++++++++

Згідно з пунктами 2 та 3 запропонованого методу, в якості поточного шару вибирається останній ($i = m - 1 = 2$), після чого для кожного ФП$_j$ в ньому знаходиться адреса $v_{2, j}$ будь-якої пустої комірки. Нехай були вибрані комірки $v_{2,1}=3, v_{2,2}=6, v_{2,3}=7, v_{2,4}=5$.

Здійснюється перехід до попереднього шару: $i = i - 1 = 1$ (пункт 4).

Відповідно до пункту 5 вибирається будь-яка з комірок з визначеним станом у ФП$_1$.
Нехай була вибрана комірка з адресою $v_{1,1}=2$, що має значення $w_{1,1}=4$.

Згідно пункту 6, за формулою $w_{i,j} = w_{i,j-1} \oplus v_{i+1,j-1}$ послідовно визначається решта значень $w_{1,j}$ в поточному (першому) шарі зліва направо, а саме: $w_{1,2} = 4 \oplus 3 = 7, w_{1,3} = 7 \oplus 6 = 1, w_{1,4} = 1 \oplus 7 = 6, w_{1,5} = 6 \oplus 5 = 3$.
При цьому у ФП$_2$, ФП$_3$ та ФП$_4$ вже є комірки, що мають таке значення $w_{1,2}=7$, $w_{1,3}=1$ та $w_{1,4}=6$, відповідно, тому адреси $v_{1,j}$ встановлюються рівними адресам цих комірок: $v_{1,2} = 1, v_{1,3} = 5, v_{1,4} = 1$. У ФП$_5$ немає комірки зі значенням $w_{1,5}=3$, тому її адреса $v_{1,5}$ вибирається випадковим чином серед пустих комірок, наприклад $v_{1,5}=6$. До ФП$_5$ записується: $f_5(6)=3$.

Так як поточний шар є першим ($i=1$), то прохід нагору завершується (згідно пункту 7).

Відповідно до 8-го пункту, компоненти вхідного вектора $X_3$ визначаються як значення на вході відповідних ФП в першому шарі: $x_{3,1}=v_{1,1}=2, x_{3,2}=v_{1,2}=1, x_{3,3}=v_{1,3}=5, x_{3,4}=v_{1,4}=1, x_{3,5}=v_{1,5}=6$.

Згідно пункту 9, для ФП$_1$ в останньому шарі випадковим чином вибирається значення комірки, адреса $v_{2,1}=3$ якої  була визначена раніше в пункті 3. Нехай це значення $w_{2,1}=6$. До ФП$_1$ записується $f_1(3)=6$.

Відповідно пункту 10, за формулою $w_{m-1,j}=(x_{q,j-1} \oplus y_{j-1}) \oplus w_{m-1,j-1}$ послідовно знаходиться решта значень комірок в останньому шарі, відповідні адреси яких були визначені раніше (в пункті 3): $w_{2,2} = (2 \oplus 2) \oplus 6 = 6, w_{2,3} = (1 \oplus 3) \oplus 6 = 4, w_{2,4} = (5 \oplus 5) \oplus 4 = 4$. Значення відповідних ФП$_j$ заповнюються наступним чином: $f_2(6)=6, f_3(7)=4, f_4(5)=4$.

Таким чином, отримано значення вектора $X_3=\{2,1,5,1,6\}$ і завершено третій прохід методу. Поточний стан ФП наведено в таблиці 1 (комірки з невизначеним станом $(-1)$ зображені як пусті комірки). %TODO ref

Так як в кожному ФП залишилося хоча б 2 комірки з невизначеним станом, то можна продовжувати формування наступних векторів $X$.

\renewcommand{\arraystretch}{1.8}
\begin{table}[h]
\begin{flushright}
	\textit{Таблиця 2: стани ФП після 3-го та 4-го проходу}
\end{flushright}
\begin{center}
\begin{tabular}{ | c | c | c | c | c | c | }
	\hline
	\# & \multicolumn{5}{ | c | }{\textbf{Номер ФП}} \\ \cline{2-6}
	& \textbf{1} & \textbf{2} & \textbf{3} & \textbf{4} & \textbf{5} \\ \hline
	\textbf{0} & & 1 & & 2 & \\ \hline
	\textbf{1} & 7 & 7 & & 6 & \\ \hline
	\textbf{2} & 4 & 5 & 5 & & 5 \\ \hline
	\textbf{3} & \textbf{6} & 2 & 4 & 3 & \\ \hline
	\textbf{4} & & & & & 7 \\ \hline
	\textbf{5} & 2 & & 1 & \textbf{4} & \\ \hline
	\textbf{6} & & \textbf{6} & 2 & & \textbf{3} \\ \hline
	\textbf{7} & & & \textbf{4} & 7 & \\ \hline	
\end{tabular}
\quad
\begin{tabular}{ | c | c | c | c | c | c | }
	\hline
	\# & \multicolumn{5}{ | c | }{\textbf{Номер ФП}} \\ \cline{2-6}
	& \textbf{1} & \textbf{2} & \textbf{3} & \textbf{4} & \textbf{5} \\ \hline
	\textbf{0} & \textbf{1} & 1 & \textbf{6} & 2 & \\ \hline
	\textbf{1} & 7 & 7 & \textbf{6} & 6 & \\ \hline
	\textbf{2} & 4 & 5 & 5 & \textbf{3} & 5 \\ \hline
	\textbf{3} & 6 & 2 & 4 & 3 & \\ \hline
	\textbf{4} & & \textbf{6} & & & 7 \\ \hline
	\textbf{5} & 2 & & 1 & 4 & \\ \hline
	\textbf{6} & & 6 & 2 & & 3 \\ \hline
	\textbf{7} & & & 4 & 7 & \\ \hline	
\end{tabular}
\end{center}
\end{table}
\renewcommand{\arraystretch}{1}

%+++++++++++++++++++++++++++++++++++++++++++++++++++++++++++++++++
\textbf{Отримання четвертого вектору $X_4$} виконується наступним чином.
%+++++++++++++++++++++++++++++++++++++++++++++++++++++++++++++++++

Згідно з пунктами 2 та 3 запропонованого методу, в якості поточного шару вибирається останній ($i = m - 1 = 2$), після чого для кожного ФП$_j$ в ньому знаходиться адреса $v_{2, j}$ будь-якої пустої комірки. Нехай були вибрані комірки $v_{2,1}=0, v_{2,2}=4, v_{2,3}=1, v_{2,4}=2$.

Здійснюється перехід до попереднього шару: $i = i - 1 = 1$ (пункт 4).

Відповідно до пункту 5 вибирається будь-яка з комірок з визначеним станом у ФП$_1$.
Нехай була вибрана комірка з адресою $v_{1,1}=2$, що має значення $w_{1,1}=2$.

Згідно пункту 6, за формулою $w_{i,j} = w_{i,j-1} \oplus v_{i+1,j-1}$ послідовно визначається решта значень $w_{1,j}$ в поточному (першому) шарі зліва направо, а саме: $w_{1,2} = 2 \oplus 0 = 2, w_{1,3} = 2 \oplus 4 = 6, w_{1,4} = 6 \oplus 1 = 7, w_{1,5} = 7 \oplus 2 = 5$.
При цьому у ФП$_2$, ФП$_4$ та ФП$_5$ вже є комірки, що мають таке значення $w_{1,2}=2$, $w_{1,4}=7$ та $w_{1,5}=5$, відповідно, тому адреси $v_{1,j}$ встановлюються рівними адресам цих комірок: $v_{1,2} = 3, v_{1,4} = 7, v_{1,5} = 2$. У ФП$_3$ немає комірки зі значенням $w_{1,3}=6$, тому її адреса $v_{1,3}$ вибирається випадковим чином серед пустих комірок, наприклад $v_{1,5}=0$. До ФП$_3$ записується: $f_3(0)=6$.

Так як поточний шар є першим ($i=1$), то прохід нагору завершується (згідно пункту 7).

Відповідно до 8-го пункту, компоненти вхідного вектора $X_4$ визначаються як значення на вході відповідних ФП в першому шарі: $x_{4,1}=v_{1,1}=5, x_{4,2}=v_{1,2}=3, x_{4,3}=v_{1,3}=0, x_{4,4}=v_{1,4}=7, x_{4,5}=v_{1,5}=2$.

Згідно пункту 9, для ФП$_1$ в останньому шарі випадковим чином вибирається значення комірки, адреса $v_{2,1}=3$ якої  була визначена раніше в пункті 3. Нехай це значення $w_{2,1}=1$. До ФП$_1$ записується $f_1(0)=1$.

Відповідно пункту 10, за формулою $w_{m-1,j}=(x_{q,j-1} \oplus y_{j-1}) \oplus w_{m-1,j-1}$ послідовно знаходиться решта значень комірок в останньому шарі, відповідні адреси яких були визначені раніше (в пункті 3): $w_{2,2} = (5 \oplus 2) \oplus 1 = 6, w_{2,3} = (3 \oplus 3) \oplus 6 = 6, w_{2,4} = (3 \oplus 5) \oplus 6 = 3$. Значення відповідних ФП$_j$ заповнюються наступним чином: $f_2(4)=6, f_3(1)=6, f_4(2)=3$.

Таким чином, отримано значення вектора $X_4=\{5,3,0,7,2\}$ і завершено четвертий прохід методу. Поточний стан ФП наведено в таблиці 2 (комірки з невизначеним станом $(-1)$ зображені як пусті комірки). %TODO ref

Так як в ФП$_3$ залишилась лише одна пуста комірка, то, згідно пункту 12, завершується формування векторів $X$.

Відповідно до пункту 13 всі невизначені значення всіх ФП заповнюються випадковими числами в діапазоні $0 \ldots 2^k-1$.

\quad

В результаті роботи методу були сформовані 4 вектори $X$: $X_1=\{1,2,3,1,2\}$, $X_2=\{2,0,6,7,4\}$, $X_3=\{2,1,5,1,6\}$ та $X_4=\{5,3,0,7,2\}$.

\section{Тестування}

Метою тестування є знаходження приблизної кількості вхідних векторів $X$, що можуть бути сформовані на основі запропонованого методу для різної довжини $n$ вихідного вектору $Y$ та розрядності фрагментів $k$.

Результаті тестування наведені у таблиці 3.

\renewcommand{\arraystretch}{1.8}
\begin{table}[h]
\begin{flushright}
	\textit{Таблиця 3: кількості сформованих вхідних векторів $X$}
\end{flushright}
\begin{center}
\begin{tabular}{ | c |  x{2.6em} | x{2.6em} | x{2.8em} | x{3.0em} | }
	\hline
	\textbf{довжина $n$} & \multicolumn{4}{ | c | }{\textbf{розрядність фрагментів $k$}} \tabularnewline \cline{2-5}
	\textbf{вихідного вектору $Y$} & \textbf{8} & \textbf{10} & \textbf{16} & \textbf{20} \tabularnewline\hline
	\textbf{80} & 50 & 262 & 26199 & 510820 \tabularnewline\hline
	\textbf{160} & 25 & 148 & 14785 & --- \tabularnewline\hline
	\textbf{240} & 16 & 104 & 10709 & --- \tabularnewline\hline
	\textbf{320} & 11 & 79 & 8537 & --- \tabularnewline\hline
\end{tabular}
\end{center}
\end{table}
\renewcommand{\arraystretch}{1}

\renewcommand{\arraystretch}{1.4}
\begin{tabular}{ | c |  x{2.6em} | x{2.6em} | x{2.6em} | x{2.6em} | x{2.8em} | x{3.0em} | x{3.0em} |}
	\hline
	\textbf{кількість $m$} & \multicolumn{7}{ | c | }{\textbf{розрядність фрагментів $k$}} \tabularnewline \cline{2-8}
	\textbf{фрагментів $Y$} & \textbf{5} & \textbf{10} & \textbf{13} & \textbf{14} & \textbf{15} & \textbf{16} & \textbf{17}  \tabularnewline\hline
	\textbf{4} & 14 & 490 & 4002 & --- & --- & 31924 & 63826 \tabularnewline\hline
	\textbf{5} & 10 & 401 & 3267 & --- & --- & 26200 & 52561 \tabularnewline\hline
	\textbf{6} & 8 & 346 & 2782 & 5608 & 11224 & 22469 & 44938 \tabularnewline\hline
	\textbf{7} & 7 & 297 & 2454 & 4919 & 9862 & 19685 & --- \tabularnewline\hline
	\textbf{8} & 5 & 262 & 2205 & 4416 & 8844 & 17698 & ---\tabularnewline\hline
	\textbf{9} & 5 & 239 & 2001 & --- & --- & 16115 & --- \tabularnewline\hline
	\textbf{10} & 4 & 223 & 1833 & --- & --- & 14785 & ---\tabularnewline\hline
\end{tabular}
\renewcommand{\arraystretch}{1}

\quad

\textbf{Найкраща спроба апроксимації:}
VectorsAmount(k, m) $ = 0.817 * 2^k * 0.8786^m$

(найбільш точна при $k \to \inf$ та $m=\{4,10\}$)

\end{document}

%+++++++++++++++++++++++++++++++++++++++++++++++++++++++++++++++++
%+++++++++++++++++++++++++++++++++++++++++++++++++++++++++++++++++
%+++++++++++++++++++++++++++++++++++++++++++++++++++++++++++++++++