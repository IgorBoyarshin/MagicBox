\documentclass[12pt]{article}
%\usepackage{amsmath}
\usepackage{mathtools}
\usepackage[a4paper, margin=1in]{geometry}
\usepackage{algpseudocode}
%\usepackage{hyperref}
\usepackage[utf8]{inputenc}
\usepackage[english, russian]{babel}
\usepackage[T1, T2A]{fontenc}


%\title{Getting started}
%\author{Veloci Raptor}
%\date{03/14/15}

\begin{document}
%\maketitle

В даній статті прийняті такі умовні позначення:
\begin{enumerate}
\item $n$ -- довжина вектора;
\item $k$ -- розрядність фрагмента;
\item $m = \frac{n}{k}$ -- кількість фрагментів;
\item $X = \{x_1, x_2, \dotsc , x_n\}$ -- вхідний вектор, $\forall i=1,2 \ldots n : x_i \in \{0,1\}$;
\item $Y = \{y_1, y_2, \dotsc , y_n\}$ -- вихідний вектор, $\forall i=1,2 \ldots n : y_i \in \{0,1\}$;
\end{enumerate}

Будь-який функціональний перетворювач (ФП) можна представити у вигляді таблиці з $2^k$ рядків. У початковому стані всі комірки заповнені значенням $-1$: $\forall i=1,2 \ldots 2m-1, j=0,1 \ldots 2^k-1 : f_i(j) = -1$.

Позначимо через $v_{i,j}$ вхідне значення до ФП i-го шару j-ї позиції, а через $w_{i,j}$ -- вихідне значення з ФП i-го шару j-ї позиції: $f_j(v_{i,j}) = w_{i,j}$.

Вихідний вектор Y вважаємо заданим.

\vspace{2em}

%Покласти $i=m-1$.

%Для всіх ФП$_j, j=1,2 \ldots 2m-1$, вибрати таке $v_{i,j}$, щоб $f_j(v_{i,j})=-1$. Якщо всі $v_{i,j}$ вибрати не вдалося $=>$ \textbf{conflict}.
Для всіх ФП$_j, j=1,2 \ldots 2m-1$, вибрати таке $v_{m-1,j}$, щоб $f_j(v_{m-1,j})=-1$. Якщо всі $v_{m-1,j}$ вибрати не вдалося $=>$ \textbf{conflict}.

\begin{algorithmic}
\For{$i=m-2,m-3 \ldots 1$}
	\State Довільним чином вибрати $v_{i,1}$, так щоб $f_1(v_{i,1}) \neq -1 $;
	\State якщо такого $v_{i,1}$ немає, то вибрати його і $w_{i,1}$ довільним чином
	\State і покласти $f_1(v_{i,1}) = w_{i,1}$.
	\For{$j=2,3 \ldots 2m-i $}
		\State $w_{i,j} = w_{i,j-1} \oplus v_{i+1,j-1}$
		\If{$i=1$ та $j>m$}
			\State Покласти $v_{i,j} = v_{i,j-m}$; якщо при цьому
			\State $f_j(v_{i,j}) \neq -1$ та $f_j(v_{i,j}) \neq w_{i,j}$ $=>$ \textbf{conflict},
			\State інакше покласти $f_j(v_{i,j}) = w_{i,j}$.
		\Else
			\State $v_{i,j}$ вибрати так, щоб $f_j(v_{i,j}) = w_{i,j}$; якщо такого немає,
			\State то вибрати його випадковим чином, щоб $f_j(v_{i,j}) = -1$,
			\State і покласти $f_j(v_{i,j}) = w_{i,j}$; якщо такого немає $=>$ \textbf{conflict}
		\EndIf
	\EndFor
\EndFor

\For{$j=1,2 \ldots 2m-1$}
	\State $x_j = v_{1,j}$
\EndFor

Вибрати $w_{m-1,1}$ довільним чином і покласти $f_1(v_{m-1,1}) = w_{m-1,1}$.

\For{$j=2,3 \ldots m$}
	\State $w_{m-1,j} = (x_j \oplus y_j) \oplus w_{m-1,j-1}$.
\EndFor
\end{algorithmic}

\vspace{2em}

Як результат роботи одного такого проходу, маємо сформований новий вектор $X$ та додатково заповнені ФП системи.

\end{document}

input\slash output

Not like this ... but like this:\\
New York, Tokyo, Budapest, \ldots
$\oplus$

Author Name \hfill \today
